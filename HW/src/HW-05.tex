\documentclass[11pt]{article}

% set these commands
\newcommand{\course}{CSCI 347}
\newcommand{\proj}{Homework 05}

\usepackage{macros}


\begin{document}

{ ~\\
    \course \\ 
    \proj \\ 
}

Show your work. Include any code snippets you used to generate an answer, using
comments in the code to clearly indicate which problem corresponds to which code

\begin{enumerate}

    \item (2 points) In Python, generate a (2-dimensional multivariate Gaussian)
    data matrix $D$ using the following code:

    \begin{verbatim}
        mu = np.array([0,0])
        Sigma = np.array([[1,0], [0, 1]])
        X1, X2 = np.random.multivariate_normal(mu, Sigma, 1000).T
        D = np.array([X1, X2]).T
    \end{verbatim} 

    Create a scatter plot of the data, with the $x$-axis corresponding to the
    first attribute (column) in $D$, and the $y$-axis corresponding to the
    second attribute (column) in $D$.
    \begin{verbatim}
        plt.scatter(D[:, 0], D[:, 1])
    \end{verbatim}
    \includegraphics{Figure_1.png}

    \item (7 points) Using the scaling matrix $S$ and rotation matrix $R$ below
    to transform the data $D$ from Question 1, by multiplying each data instance
    (row) $x_i$ by $RS$. Let $D_{RS}$ be the matrix of
    the transformed data. That is, each 2-dimensional row vector $x_i$
    in $D$ should be transformed into a 2-dimensional vector $RSx_i$ in $D_{RS}$.

    \begin{enumerate}

        \item (4 points) Plot the transformed data $D_{RS}$ in the same figure
        as the original data $D$, using different colors to differentiate
        between the original and transformed data.

        $$
            R = \begin{pmatrix}
                \cos(\pi/4) & -\sin(\pi/4) \\
                \sin(\pi/4) & \cos(\pi/4)
            \end{pmatrix},
            \qquad
            S = \begin{pmatrix}
                5 & 0 \\
                0 & 2
            \end{pmatrix}
        $$
        \includegraphics{Figure_2.png}
        \newpage
        \item (2 points) Write down the covariance matrix of the transformed
        data $D_{RS}$.
        \newline \begin{verbatim}
        Drs covariance matrix =
            [[ 1.55653365e+01 -1.28722821e+01  4.13750509e+00 ...  7.37080401e-01
              -2.39302940e+01 -4.55958784e+00]
            [-1.28722821e+01  1.06451696e+01 -3.42164994e+00 ... -6.09553597e-01
              1.97899670e+01  3.77070556e+00]
            [ 4.13750509e+00 -3.42164994e+00  1.09981229e+00 ...  1.95927271e-01
              -6.36103903e+00 -1.21200835e+00]
            ...
           [ 7.37080401e-01 -6.09553597e-01  1.95927271e-01 ...  3.49036795e-02
              -1.13319430e+00 -2.15914563e-01]
           [-2.39302940e+01  1.97899670e+01 -6.36103903e+00 ... -1.13319430e+00
              3.67906578e+01  7.00995300e+00]
           [-4.55958784e+00  3.77070556e+00 -1.21200835e+00 ... -2.15914563e-01
              7.00995300e+00  1.33564997e+00]]
        \end{verbatim}

        \item (1 point) What is the total variance of the transformed data
        $D_{RS}$.
        \newline Drs total variance =  7.349058187383844
    \newline Question 2 Code:    
    \begin{verbatim}
        
R = np.array([[0.707, -0.707],
     [0.707, 0.707]])

S = np.array([[5, 0],
     [0, 2]])

RS = R * S


D_rs = np.empty([1000, 2])
for i in range(len(D)):
    D_rs[i, 0] = (D[i, 0] * RS[0,0]) + (D[i, 1] * RS[0,1])
    D_rs[i, 1] = (D[i, 0] * RS[1,0]) + (D[i, 1] * RS[1,1])

print("D_rs covariance matrix = ", np.cov(D_rs))
print("D_rs total variance = ", np.var(D_rs))

plt.scatter(D[:, 0], D[:, 1])
plt.scatter(D_rs[:, 0], D_rs[:, 1], color = "green")
plt.show()

    \end{verbatim}
    \newpage
    \end{enumerate}

    \item (8 points) Use sklearn’s PCA function to transform the data matrix
    $D_{RS}$ from Question 2 to a 2-dimensional space where the coordinate axes
    are the principal components.

    \begin{enumerate}

        \item (4points) Plot the PCA-transformed data, with the $x$-axis
        corresponding to the first principal component and the $y$-axis
        corresponding to the second principal component.
        \newline \includegraphics{Figure_3.png}

        \item (2 points) What is the estimated covariance matrix of the
        PCA-transformed data?
         \begin{verbatim}
D_pca covariance matrix =  [[  9.7526634   -2.11908641  11.2666192  ...   1.58981398  -4.4306419
  -16.70136789]
 [ -2.11908641   0.46044112  -2.44804303 ...  -0.3454393    0.96270246
    3.62892067]
 [ 11.2666192   -2.44804303  13.01559409 ...   1.83660894  -5.11843309
  -19.2940066 ]
 ...
 [  1.58981398  -0.3454393    1.83660894 ...   0.25916085  -0.72225362
   -2.72254533]
 [ -4.4306419    0.96270246  -5.11843309 ...  -0.72225362   2.01284375
    7.58744327]
 [-16.70136789   3.62892067 -19.2940066  ...  -2.72254533   7.58744327
   28.60097574]]

        \end{verbatim}

        \item (2 points) What is the fraction of the total variance captured in
        the direction of the first principal component? What is the fraction of
        the total variance captured in the direction of the second principal
        component?
        \newline Fraction of variance captured by first principal component:  0.8654426212621087
        \newline Fraction of variance captured by first principal component:  0.1345573787378913 
        
        Question 3 Code:
        \begin{verbatim}
pca = PCA(n_components = 2)
D_pca = pca.fit_transform(D_rs)

print("D_pca covariance matrix = ", np.cov(D_pca))
print("Fraction of variance captured by first principal component: "
      , pca.explained_variance_ratio_[0])
print("Fraction of variance captured by first principal component: "
      , pca.explained_variance_ratio_[1])


plt.scatter(D_pca[:, 0], D_pca[:, 1], color = "orange")
plt.show()

            
        \end{verbatim}

    \end{enumerate}

    \newpage
    \item (18 points) Load the Boston data set into Python using sklearn's
    datasets package. Use sklearn's PCA function to reduce the dimensionality of
    the data to 2 dimensions.

    
    \begin{enumerate}

        \item (5 points) First, standard-normalize the data. Then, create a
        scatter plot of the 2-dimensional, PCA-transformed normalized Boston
        data, with the $x$-axis corresponding to the first principal component
        and the $y$-axis corresponding to the second principal component.
        \label{item:4.1}
        \newline \includegraphics{Figure_4.png}
        \newline \begin{verbatim}
boston = load_boston()['data']

multiMean = np.mean(boston, axis = 0)
bostonNormed = boston - multiMean
pca = PCA(n_components = 13)
Boston_pca = pca.fit_transform(bostonNormed)

plt.scatter(Boston_pca[:, 0], Boston_pca[:, 1], color = "teal")
plt.show()
        \end{verbatim}
        \newpage
        \item (3 points) Create a plot of the fraction of the total variance
        explained by the first $r$ components for $r = 1, 2, \ldots, 13$.
        \newline \includegraphics{Figure_5.png}
        \newline \begin{verbatim}
var_ratio = pca.explained_variance_ratio_
plt.plot(range(1,len(var_ratio)+1), np.cumsum(var_ratio), marker='x')
plt.title('Iris data: fraction of total variance preserved by principal components')
plt.xlabel('r : the number of principal components')
plt.ylabel('f(r) : fraction of total variance preserved')
plt.show()
        \end{verbatim}

        \item (2 points)
        \begin{enumerate}

            \item (1 point) If we want to capture at least 90\% of the variance
            of the normalized Boston data, how many principal components (i.e.,
            what dimensionality) should we use?
            \newline We should use at least two principal components if we want to capture at least 90 percent of the variance per above plot.
            

            \item (1 point) If we use two principal components of the normalized
            Boston data, how much (what fraction or percentage) of the total
            variance do we capture?
            \newline We capture 96.9 percent of the total variance by using two principal components.

        \end{enumerate}

        \item (4 points) Use scikit-learn's implementation of $k$-means to find
        2 clusters in the two-dimensional, PCA-transformed normalized Boston
        data set (the input to $k$-means should be the data that was plotted in
        part \ref{item:4.1}). Plot the 2-dimensional data with colors corresponding to
        predicted cluster membership for each point. On the same plot, also plot
        the two means found by the $k$-means algorithm in a different color than
        the colors used for the data.
        \newline \includegraphics{Figure_6.png}
        \newline \begin{verbatim}
kmeans = KMeans(n_clusters=2, random_state=0).fit(Boston_pca)

centers = kmeans.cluster_centers_
labels = kmeans.fit_predict(Boston_pca)
plt.scatter(Boston_pca[:, 0], Boston_pca[:, 1], c = labels)
plt.scatter(centers[:, 0], centers[:, 1], c = 'red')
plt.show()
        \end{verbatim}
        \newpage
        \item (4 points) Use scikit-learn's implementation of DBSCAN to find
        clusters in the two-dimensional, PCA-transformed normalized Boston data
        set (the input to DBSCAN should be the data that was plotted in part
        \label{item:4.1}). Plot the 2-dimensional data with colors corresponding
        to predicted cluster membership for each point. Noise points should be
        colored differently than any of the clusters. How many clusters were
        found by DBSCAN?
        \newline \includegraphics{Figure_7.png} 
        \newline \begin{verbatim}
dbs = DBSCAN(eps=.6, min_samples=5).fit(Boston_pca)
labels = dbs.fit_predict(Boston_pca)

plt.scatter(Boston_pca[:, 0], Boston_pca[:, 1], c = labels)

plt.show()
        \end{verbatim}
        This has three clusters: Cyan, Blue and Yellow

    \end{enumerate}

\end{enumerate}

{\bf Acknowledgements:} Homework problems adapted from assignments of
Veronika Strnadova-Neeley.

\end{document}
