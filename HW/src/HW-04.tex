\documentclass[11pt]{article}

% set these commands
\newcommand{\course}{CSCI 347}
\newcommand{\proj}{Homework 04}

\usepackage{macros}


\begin{document}

{ ~\\
    \course \\ 
    \proj \\ 
}

Show your work. Include any code snippets you used to generate an answer, using
comments in the code to clearly indicate which problem corresponds to which code

\begin{enumerate}

    \item (2 points) Consider matrix $A$ and vector $v$.
    Compute the matrix-vector product $Av$.
    $$
        A = \begin{pmatrix} 2 & 1 \\ 1 & 3 \end{pmatrix},
        v = \begin{pmatrix}-1 & 1\end{pmatrix}
    $$
    \newline
    \newline $ Av = \begin{pmatrix} 2*(-1)+1(1) \\ 1*(-1) + 3(1) \end{pmatrix}$ $ Av = \begin{pmatrix} -1 \\ 2 \end{pmatrix}$


    \item Consider matrix $A$ and data set $D$:
    $$
        A = \begin{pmatrix}
            \frac{\sqrt{3}}{2} & -\frac{1}{2} \\ 
            \frac{1}{2} & \frac{\sqrt{3}}{2}
        \end{pmatrix},
        D = \begin{pmatrix}
            1  &  1.5 \\
            1  &  2 \\
            3  &  4 \\
            -1 &  -1 \\
            -1 &  1 \\
            1  & -2 \\
            2  &  2 \\
            2  & 3
        \end{pmatrix}
    $$

    \begin{enumerate}

        \item (2 points) Let $X_1$ and $X_2$ be the first and second attributes
        of the data, respectively.  Use Python to create a scatter plot of the
        data, where the $x$-axis is $X_1$ and the $y$-axis is~$X_2$.
        \newline\begin{lstlisting} import pandas as pd
        import matplotlib.pyplot as plt
        
        data = [(1, 1.5),
                (1, 2),
                (3, 4),
                (-1, -1),
                (-1, 1),
                (1, -2),
                (2, 2),
                (2, 3)]
        
        df = pd.DataFrame(data)
        plt.scatter(df[0], df[1])
        plt.show()
        
        \end{lstlisting}
        \newline \includegraphics{scatter1.png}

        \item (4 points) Treating each row as a 2-dimensional vector, apply the
        linear transformation $A$ to each row.  In other words, let $x_i$ be the
        $i$-th row of $D$. For each $x_i$, find the matrix-vector product $A
        x_i$.  For example, $x_2 = \begin{pmatrix} 1 \\  2 \end{pmatrix}$.
        \newline \begin{lstlisting}
        D = np.array([(1, 1.5),
        (1, 2),
        (3, 4),
        (-1, -1),
        (-1, 1),
        (1, -2),
        (2, 2),
        (2, 3)])
        
        A = np.array([((math.sqrt(3)/2), (-1/2)),
                    ((1/2), (math.sqrt(3)/2))])
                    
        linear_transformed_data = A.dot(D.T)
        \end{lstlisting}
        
        $$ transformed = \begin{pmatrix}
         0.1160254  & 1.79903811 \\
         -0.1339746  & 2.23205081 \\
          0.59807621 & 4.96410162 \\
         -0.3660254 & -1.3660254  \\
         -1.3660254 & 0.3660254  \\
          1.8660254 & -1.23205081 \\
          0.73205081 & 2.73205081 \\
          0.23205081 & 3.59807621
         \end{pmatrix}$$

        \item (3 points) Use Python to create a plot showing both the original
        data and the transformed data, with the $x$-axis still corresponding to
        $X_1$ and the y-axis corresponding to $X_2$. Use different colors and
        markers to differentiate between the original and transformed data. That
        is, each transformed data point in the plot should be one matrix-vector
        product $Ax_i$, which is a 2-dimensional vector. Each original point in
        the plot should have the same coordinates as it did in part 2.1.
        \newline \begin{lstlisting}
        fig = plt.figure()
        ax = fig.add_subplot(1, 1, 1)
        ax.scatter(D[:, 0], D[:, 1], color = 'blue')
        ax.scatter(L.T[:, 0], L.T[:, 1], color = 'green', marker = '+')
        plt.show()
        \end{lstlisting}
        \newline \includegraphics{scatter2.png}
        \item (1 point) Write down the multi-dimensional mean of the data.
        (Remember that this should be a 2-dimensional vector)
        \newline \begin{lstlisting}
        mm = np.mean(D, axis = 0)
        mmL1 = L.T[:, 0].sum() / 8
        mmL2 = L.T[:, 1].sum() / 8
        \end{lstlisting}
        \newline $$ multi Mean = \begin{pmatrix}
        1 & 1.3125 \\
        0.2098 & 1.6367 \\
        \end{pmatrix}$$
        \newpage
        \item (2 points) Mean-center the data. Write down the mean-centered data
        matrix.
        \newline \begin{lstlisting}
        centering = lambda x: x- x.mean()
        Z = centering(L)
        print(Z)
        print(Z.mean()) #extremely close to 0
        \end{lstlisting}
        \newline \includegraphics{centered.png}

        \item (2 points] Use Python to create a scatter plot showing both the
        original data and the mean-centered data, where the $x$-axis is $X_1$
        and the $y$-axis is $X_2$. Use different colors and markers to
        differentiate between the original and mean-centered data.
        \newline \begin{lstlisting}
        fig = plt.figure()
        ax = fig.add_subplot(1, 1, 1)
        ax.scatter(Z.T[:, 0], Z.T[:, 1], color = 'red', marker = '_')
        ax.scatter(L.T[:, 0], L.T[:, 1], color = 'green', marker = '+')
        plt.show()
        \end{lstlisting}
        \newline \includegraphics{scatter3.png}
        \newpage
        \item (3 points) Write down the covariance matrix of the data matrix
        $D$. Use estimated covariance.
        \newline print(np.cov(D))
        \newline \includegraphics{cov.png}
        
        \item (3 points) Write down the covariance matrix of the centered data
        matrix $Z$. Use estimated covariance.
        \newline print(np.cov(Z))
        $$ cov(Z) = \begin{pmatrix}
            0.87269389 & 0.0954086 \\
            0.0954086 & 5.05141325
        \end{pmatrix}$$

        \item (3 points) Write down the covariance matrix of the data after
        applying standard normalization.
        \newline \begin{lstlisting}
            Z_normalized = MinMaxScaler().fit_transform(Z)
            print(np.cov(Z_normalized))
            \end{lstlisting}
        \newline $$cov(Z_normalized) = \begin{pmatrix}
        0.21428571 & -0.21428571 \\
        -0.21428571 & 0.21428571
        \end{pmatrix}$$
    \end{enumerate}

\end{enumerate}

{\bf Acknowledgements:} Homework problems adapted from assignments of
Veronika Strnadova-Neeley.

\end{document}